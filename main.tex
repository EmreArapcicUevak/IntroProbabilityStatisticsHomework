\documentclass[a4paper, 10pt]{article}

\usepackage[margin = 1in]{geometry} % for spacing around
\usepackage{graphicx} % for including images in your pdfs
\usepackage{xcolor} % for including colors in your pdf
\usepackage{soul} % for text decoration
\usepackage[utf8]{inputenc} % for encoded text
\usepackage[T1]{fontenc}
\usepackage{setspace} % for setting different line spacings between paragrafs.
\usepackage{enumerate} % for letting us get more detailed enumerate lists
\usepackage{multirow} % to let us combine more rows together
\usepackage{colortbl} % for decorating tables
\usepackage{amsmath} % used for representing more complicated math displays
\usepackage{supertabular}
\usepackage{longtable} % both of these packages are used to making really big tables
\usepackage{wrapfig} % allows us to wrap text around figures
\usepackage{fancyhdr} % for making fancy headers
%\usepackage{bibtex} % for making better bibliographies
\usepackage[pdftex]{hyperref} % for letting us make links
\usepackage{lscape} % Allows us to flip from portrait to landspace
\usepackage{tikz} % for high detailed drawing
\usepackage{multicol} % To put things side by side
\usepackage{rotating} % For rotating objects
% \usepackage{draftwatermark} % For adding watermarks
\usepackage{MnSymbol} % for using multiple symbols
\usepackage{mathtools} % Used for more math symbols
\usepackage{xfrac} % For more complciated fractions and to add derivitives
\usepackage{enumitem} % for better enum lists
\usepackage{tcolorbox} % for adding colored text boxes
\usepackage{bm} % Adding bold text to math inputs
\usepackage{pgfplots} % Used for plotting functions
\usepackage{background}
\usepackage{cancel} % For cancelling terms visually
\usepackage{microtype} % Improves spacing
\usepackage{booktabs} % For nicer tables
\usepackage{parskip}
\usepackage{array}

% Setting up the default image path
\graphicspath{{./Images/}}

% Implementing authro details
\title{Understanding Probability and Statistics: Homework Solutions}
\author{Emre Arapcic-Uevak}
\date{}

% Setting up the fancy page style
\fancypagestyle{customStyle}{
	\lhead{} \chead{} \rhead{}
	\lfoot{} \cfoot{\thepage} \rfoot{}
	\renewcommand{\headrulewidth}{0pt}
	\renewcommand{\footrulewidth}{1pt}
}
\pagestyle{customStyle}

% Settings for the background package
\backgroundsetup{
    scale=1, % Size - adjust as necessary
    color=black, % Color of the logo
    opacity=0.1, % Opacity - adjust to your liking; 0 = fully transparent, 1 = fully opaque
    angle=0, % Rotation angle if you want to rotate the logo
    position=current page.center, % Position - you can change this as necessary
    vshift=0cm, % Vertical shift - adjust as necessary
    hshift=0cm, % Horizontal shift - adjust as necessary
    contents={\includegraphics[width=10cm]{Images/Logo}} % Replace 'your_logo_filename' with the name of your logo file
}

% Setting up hyperref options
\hypersetup {
	colorlinks = false,
	citecolor = black,
	filecolor = blue,
	linkcolor = blue,
	urlcolor = blue,
	pdftex
}

% Custom commands


\begin{document}
	\maketitle
	\vspace{5mm}
	
	\begin{abstract}
		\begin{center}
			\noindent \"Understanding Probability and Statistics: Homework Solutions\" is a straightforward and accessible document designed to assist students in grasping the fundamental concepts of probability and statistics. This assignment blends traditional hand-written problem-solving methods with basic Python programming to offer a practical approach to learning. It covers a variety of problems, from simple probability calculations to more intricate statistical analyses. Each question is addressed with clear, concise solutions, demonstrating the process step-by-step, both on paper and through Python code. This approach not only helps in understanding the theoretical aspects but also in applying these concepts practically. The document serves as an essential tool for students looking to strengthen their foundation in probability and statistics through hands-on learning and problem-solving.
		\end{center}
	\end{abstract}
	\pagebreak
	
	\tableofcontents
	\pagebreak

    \section{Superhero Problem}
        \subsection{Problem Text}
            \noindent In a world of superheroes, there are 10 superheroes with unique powers, and you need to form a superhero team. Your superhero team will consist of 6 members selected from these 10 superheroes. Each superhero has their own special abilities.

            \begin{enumerate}[label=(\alph*)]
                \item Calculate the total number of unique ways to form a superhero team of 6 members from the 10 superheroes.
                \item Each superhero has an associated color: Red, Blue, Green, Yellow, Purple, Orange, Pink, Black, White, and Silver. Calculate the probability that your superhero team, randomly selected, will have at least one member with a Red, Blue, and Green color-related power.
                \item Among the 10 superheroes, there are 4 heroes with elemental powers (Earth, Water, Fire, Air) and 6 heroes with technology-based powers (Robotics, Hacking, Gadgets, Invention, Nanotechnology, and AI). What is the probability that your superhero team will consist of 3 heroes with elemental powers and 3 heroes with technology-based powers?
                \item Three of the superheroes are known for their incredible strength, while the others have different abilities. Calculate the probability that your superhero team will have exactly 2 of the 3 strongest superheroes.
                \item Two of the superheroes have a long history of teaming up on missions. What is the probability that both of them are selected for your new superhero team of 6 members?
            \end{enumerate}

        \subsection{Solutions}

            \begin{enumerate}[label=(\alph*)]
                \item
                \[
                    \begin{aligned}
                        \binom{10}{6} &= \frac{10!}{6!(10-6)!} \\
                        &= \frac{10 \cdot 9 \cdot 8 \cdot 7 \cdot \cancel{6!}}{\cancel{6!} \cdot 4 \cdot 3 \cdot 2 \cdot 1} \\
                        &= \frac{10 \cdot 9 \cdot 8 \cdot 7}{4 \cdot 3 \cdot 2 \cdot 1} \\
                        &= \boxed{210}
                    \end{aligned}
                \]

                \item
                \[
                    \begin{aligned}
                        P(\text{at least one r,b,g}) &= 1 - P(\text{no r,b,g}) \\
                        &= 1 - \left( \frac{\binom{7}{6}}{\binom{10}{6}} \right) \\
                        &= 1 - \left( \frac{7}{210} \right) \\
                        &= 1 - \frac{1}{30} \\
                        &= \frac{29}{30} \\
                        &= \boxed{0.9667}
                    \end{aligned}
                \]

                \item
                \[
                    \begin{aligned}
                        P_{3 \text{ elemental,3 technology}} &= \left( \frac{\binom{4}{3} \cdot \binom{6}{3}}{\binom{10}{6}} \right) \\
                        &= \frac{4 \cdot \frac{6!}{3! \cdot 3!}}{210} \\
                        &= \frac{4 \cdot 20}{210} \\
                        &= \frac{80}{210} \\
                        &= \frac{8}{21} \\
                        &= \boxed{0.38}
                    \end{aligned}
                \]

                \item
                \[
                    \begin{aligned}
                        P_{2 \text{ of 3 strongest}} &= \left( \frac{\binom{3}{2} \cdot \binom{7}{4}}{\binom{10}{6}} \right) \\
                        &= \frac{3 \cdot \frac{7!}{4! \cdot 3!}}{210} \\
                        &= \frac{3 \cdot 35}{210} \\
                        &= \frac{105}{210} \\
                        &= \frac{1}{2} \\
                        &= \boxed{0.5}
                    \end{aligned}
                \]

                \item
                \[
                    \begin{aligned}
                        P_{\text{both selected}} &= \left( \frac{\binom{2}{2} \cdot \binom{8}{4}}{\binom{10}{6}} \right) \\
                        &= \frac{1 \cdot \frac{8!}{4! \cdot 4!}}{210} \\
                        &= \frac{70}{210} \\
                        &= \frac{1}{3} \\
                        &= \boxed{0.3333}
                    \end{aligned}
                \]
            \end{enumerate}

    \pagebreak
    
    \section{Adventurer Problem}
        \subsection{Problem Text}
            \noindent Imagine you are an adventurer on a quest to discover the fabled Enchanted Treasure Chests hidden deep in a mysterious forest. As you embark on your adventure, you come across three magical treasure chests, each guarded by a forest spirit. You believe that one of these chests contains the legendary Enchanted Jewel, while the other two chests hold only rocks. Here's how your quest unfolds:

            \begin{itemize}
                \item The three treasure chests are placed before you, each with a unique mystical symbol: Chest X, Chest Y, and Chest Z. The Enchanted Jewel is hidden in one of these chests, but you have no idea which one.
                \item You must make an initial choice and select one of the three chests. This will be your starting selection.
                \item The forest spirits, knowing the contents of each chest, will then reveal one of the two remaining chests that does not contain the Enchanted Jewel. They always choose one of the two non-Jewel chests to reveal.
                \item You are now faced with a decision. You can either stick with your initial choice or switch to the other unopened chest.
            \end{itemize}

            What is the probability of discovering the Enchanted Jewel if you choose to switch chests, and what is the probability of finding it if you stick with your initial selection? Provide a detailed explanation of your answer, including any probability calculations. Please remember to show your work and offer a clear rationale for your response.

        \subsection{Solution}
            \subsubsection{Probability Calculation}
                Given three chests X, Y, and Z, the probabilities are as follows:

                \begin{align*}
                    P(X) &= \frac{1}{3} \\
                    P(Y \text{ or } Z) &= \frac{2}{3}
                \end{align*}

                \noindent Spirit choices:
                \begin{align*}
                    P(Y \text{ or } Z | X) &= \frac{1}{2} \quad \text{if I choose chest with jewel (X)} \\
                    P(Z | Y) &= 1 \quad \text{if I choose chest with rock}
                \end{align*}

                \noindent Case 1: Sticking with initial choice
                \begin{align*}
                    P(\text{Jewel} | \text{initial choice}) &= P(X) \cdot \left(1 - P(Y \text{ or } Z | X)\right) \\
                    &= \frac{1}{3} \cdot \left(1 - \frac{1}{2}\right) \\
                    &= \frac{1}{3} \cdot \frac{1}{2} \\
                    &= \frac{1}{6}
                \end{align*}

                \noindent Case 2: Switching to other chest
                \begin{align*}
                    P(\text{Jewel} | \text{switch}) &= P(Y \text{ or } Z) \cdot P(Z | Y) \\
                    &= \frac{2}{3} \cdot 1 \\
                    &= \frac{2}{3}
                \end{align*}

            \subsubsection{Explanation}
                \begin{enumerate}
                    \item \textbf{Initial Choice:} The probability of selecting the chest with the jewel upon the initial choice is \( \frac{1}{3} \), as each of the three chests has an equal chance of containing the jewel. Assuming I initially choose the chest with the jewel, the spirit then has a \( \frac{1}{2} \) probability of opening one of the other two chests. Consequently, the probability of obtaining the jewel by sticking with the initial choice is \( \frac{1}{6} \), which is the product of the initial correct choice and the spirit's subsequent action.

                    \item \textbf{Switching Choice:} If my initial choice is incorrect, which occurs with a probability of \( \frac{2}{3} \), the jewel must be in one of the other two chests. Since the spirit will always reveal a chest without the jewel, the remaining unopened chest has a \( \frac{2}{3} \) chance of containing the jewel. Thus, switching the choice leads to a \( \frac{2}{3} \) probability of finding the jewel, reflecting the combined likelihood of the initial wrong choice and the guaranteed action of the spirit revealing a non-jewel chest.
                \end{enumerate}

    \section{Business Strategy Dilemma}
        \subsection{Problem Text}
            \noindent As a coffee shop owner in a small town, you are faced with a significant decision regarding your business strategy, and it’s crucial to make an informed choice. Your competitor, another coffee shop owner in the town, is also pondering the same decision. Here’s the situation:

            \begin{itemize}
                \item If both you and your competitor choose to expand your businesses, there will be increased competition, resulting in a slight decrease in profits for both of you.
                \item If one of you chooses to expand while the other chooses to maintain the status quo, the one who expands will gain significant profits, while the other will lose some business but maintain a decent profit.
                \item If both you and your competitor choose to maintain the status quo, you will continue to earn your usual profits without any additional costs.
                \item Neither you nor your competitor can communicate during this process, and your decisions are independent of each other.
            \end{itemize}

            \noindent What strategy would you select in this business scenario, and what factors influence your choice? How do you predict your competitor will decide, and what outcomes do you anticipate for both of your coffee shops? Please provide your reasoning, considering the potential consequences of your choices in this business context.
    
        \subsection{Solution}
            In analyzing the strategic decision to expand a coffee shop business, we will be using game theory; we consider two players in the game: You (Player A) and your competitor (Player B). Each player has two strategies: to expand (E) or maintain status quo (S). The payoffs are as follows:

            \begin{itemize}
                \item Both players expanding results in increased competition, denoted as a slight decrease in profits (\(-\)).
                \item One player expanding while the other does not result in significant profit for the expander (\(++\)) and a slight decrease for the non-expander (\(-\)).
                \item Both players maintaining status quo results in the usual profit (0) for both.
            \end{itemize}

            The payoff matrix for this game can be represented as:

            \begin{center}
                \begin{tabular}{cc|c|c|}
                    \cline{3-4}
                    & & \multicolumn{2}{ c| }{Player B} \\ \cline{3-4}
                    & & E & S \\ \cline{1-4}
                    \multicolumn{1}{ |c  }{\multirow{2}{*}{Player A} } &
                    \multicolumn{1}{ |c| }{E} & \(-, -\) & \(++, -\) \\ \cline{2-4}
                    \multicolumn{1}{ |c  }{}                        &
                    \multicolumn{1}{ |c| }{S} & \(-, ++\) & \(0, 0\) \\ \cline{1-4}
                \end{tabular}
            \end{center}

            Analyzing the payoff matrix, we see that if Player A expands (E), the best response for Player B is also to expand (E), leading to a Nash Equilibrium at (E, E). The same applies if Player A chooses to maintain status quo (S). Therefore, there are two Nash Equilibrium in this game: (E, E) and (S, S), indicating that both players will either expand or not expand.

            Given the potential for higher profits, a rational player would choose to expand. However, this scenario assumes that both players are rational and have the same payoff values. In real-world scenarios, other factors such as risk preference, market capacity, and financial constraints could influence the decision.
            In our game-theoretical approach, we operate under the assumption that if one player chooses to expand and the other does not, the non-expander will incur a greater loss than if both players had chosen to expand. This is a critical consideration, as it influences the identification of Nash Equilibria within the model. However, there is an additional strategic element to consider. By allowing the competitor to expand without opposition, the non-expander not only sustains immediate financial losses but also risks yielding significant market share to the competitor. This could potentially weaken their competitive position in the long term. Conversely, the decision not to expand, while resulting in short-term financial disadvantage, may enable a business to capitalize on future opportunities and overtake the competition as circumstances evolve. It is imperative, therefore, to balance the immediate economic implications against the long-term strategic outlook, recognizing that ceding ground in the short term might offer tactical advantages later on.

\end{document}
