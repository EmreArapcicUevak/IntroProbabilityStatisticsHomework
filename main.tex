\documentclass[a4paper, 10pt]{article}

\usepackage[margin = 1in]{geometry} % for spacing around
\usepackage{graphicx} % for including images in your pdfs
\usepackage{xcolor} % for including colors in your pdf
\usepackage{soul} % for text decoration
\usepackage[utf8]{inputenc} % for encoded text
\usepackage[T1]{fontenc}
\usepackage{setspace} % for setting different line spacings between paragrafs.
\usepackage{enumerate} % for letting us get more detailed enumerate lists
\usepackage{multirow} % to let us combine more rows together
\usepackage{colortbl} % for decorating tables
\usepackage{amsmath} % used for representing more complicated math displays
\usepackage{supertabular}
\usepackage{longtable} % both of these packages are used to making really big tables
\usepackage{wrapfig} % allows us to wrap text around figures
\usepackage{fancyhdr} % for making fancy headers
%\usepackage{bibtex} % for making better bibliographies
\usepackage[pdftex]{hyperref} % for letting us make links
\usepackage{lscape} % Allows us to flip from portrait to landspace
\usepackage{tikz} % for high detailed drawing
\usepackage{multicol} % To put things side by side
\usepackage{rotating} % For rotating objects
% \usepackage{draftwatermark} % For adding watermarks
\usepackage{MnSymbol} % for using multiple symbols
\usepackage{mathtools} % Used for more math symbols
\usepackage{xfrac} % For more complciated fractions and to add derivitives
\usepackage{enumitem} % for better enum lists
\usepackage{tcolorbox} % for adding colored text boxes
\usepackage{bm} % Adding bold text to math inputs
\usepackage{pgfplots} % Used for plotting functions
\usepackage{background}

% Setting up the default image path
\graphicspath{{./Images/}}

% Implementing authro details
\title{Understanding Probability and Statistics: Homework Solutions}
\author{Emre Arapcic-Uevak}
\date{}

% Setting up the fancy page style
\fancypagestyle{customStyle}{
	\lhead{} \chead{} \rhead{}
	\lfoot{} \cfoot{\thepage} \rfoot{}
	\renewcommand{\headrulewidth}{0pt}
	\renewcommand{\footrulewidth}{1pt}
}
\pagestyle{customStyle}

% Settings for the background package
\backgroundsetup{
    scale=1, % Size - adjust as necessary
    color=black, % Color of the logo
    opacity=0.1, % Opacity - adjust to your liking; 0 = fully transparent, 1 = fully opaque
    angle=0, % Rotation angle if you want to rotate the logo
    position=current page.center, % Position - you can change this as necessary
    vshift=0cm, % Vertical shift - adjust as necessary
    hshift=0cm, % Horizontal shift - adjust as necessary
    contents={\includegraphics[width=10cm]{Images/Logo}} % Replace 'your_logo_filename' with the name of your logo file
}

% Setting up hyperref options
\hypersetup {
	colorlinks = false,
	citecolor = black,
	filecolor = blue,
	linkcolor = blue,
	urlcolor = blue,
	pdftex
}

% Custom commands


\begin{document}
	\maketitle
	\vspace{5mm}
	
	\begin{abstract}
		\begin{center}
			\noindent \"Understanding Probability and Statistics: Homework Solutions\" is a straightforward and accessible document designed to assist students in grasping the fundamental concepts of probability and statistics. This assignment blends traditional hand-written problem-solving methods with basic Python programming to offer a practical approach to learning. It covers a variety of problems, from simple probability calculations to more intricate statistical analyses. Each question is addressed with clear, concise solutions, demonstrating the process step-by-step, both on paper and through Python code. This approach not only helps in understanding the theoretical aspects but also in applying these concepts practically. The document serves as an essential tool for students looking to strengthen their foundation in probability and statistics through hands-on learning and problem-solving.
		\end{center}
	\end{abstract}
	\pagebreak
	
	\tableofcontents
	\pagebreak

    \section{Superhero Problem}
        \subsection{Problem Text}
            \noindent In a world of superheroes, there are 10 superheroes with unique powers, and you need to form a superhero team. Your superhero team will consist of 6 members selected from these 10 superheroes. Each superhero has their own special abilities.

            \begin{enumerate}[label=(\alph*)]
                \item Calculate the total number of unique ways to form a superhero team of 6 members from the 10 superheroes.
                \item Each superhero has an associated color: Red, Blue, Green, Yellow, Purple, Orange, Pink, Black, White, and Silver. Calculate the probability that your superhero team, randomly selected, will have at least one member with a Red, Blue, and Green color-related power.
                \item Among the 10 superheroes, there are 4 heroes with elemental powers (Earth, Water, Fire, Air) and 6 heroes with technology-based powers (Robotics, Hacking, Gadgets, Invention, Nanotechnology, and AI). What is the probability that your superhero team will consist of 3 heroes with elemental powers and 3 heroes with technology-based powers?
                \item Three of the superheroes are known for their incredible strength, while the others have different abilities. Calculate the probability that your superhero team will have exactly 2 of the 3 strongest superheroes.
                \item Two of the superheroes have a long history of teaming up on missions. What is the probability that both of them are selected for your new superhero team of 6 members?
            \end{enumerate}

        \subsection{Solutions}

            \begin{enumerate}[label=(\alph*)]
                \item
                    \begin{align*}
                        \binom{10}{6} &= \frac{10!}{6!(10-6)!} \\
                        &= \frac{10 \cdot 9 \cdot 8 \cdot 7 \cdot 6!}{6! \cdot 4!} \\
                        &= \frac{10 \cdot 9 \cdot 8 \cdot 7}{4 \cdot 3 \cdot 2 \cdot 1} \\
                        &= 210
                    \end{align*}

                \item
                    \begin{align*}
                        P(\text{at least one r,b,g}) &= 1 - P(\text{no r,b,g}) \\
                        &= 1 - \left( \frac{\binom{7}{6}}{\binom{10}{6}} \right) \\
                        &= 1 - \left( \frac{7}{210} \right) \\
                        &= 1 - \frac{1}{30} \\
                        &= \frac{29}{30} \\
                        &= 0.9667
                    \end{align*}

                \item
                    \begin{align*}
                        P(3 \text{ elemental,3 technology}) &= \left( \frac{\binom{4}{3} \cdot \binom{6}{3}}{\binom{10}{6}} \right) \\
                        &= \frac{4 \cdot \frac{6!}{3!3!}}{210} \\
                        &= \frac{4 \cdot 20}{210} \\
                        &= \frac{80}{210} \\
                        &= \frac{8}{21} \\
                        &= 0.38
                    \end{align*}

                \item
                    \begin{align*}
                        P(2 \text{ of 3 strongest}) &= \left( \frac{\binom{3}{2} \cdot \binom{7}{4}}{\binom{10}{6}} \right) \\
                        &= \frac{3 \cdot \frac{7!}{4!3!}}{210} \\
                        &= \frac{3 \cdot 35}{210} \\
                        &= \frac{105}{210} \\
                        &= \frac{1}{2} \\
                        &= 0.5
                    \end{align*}

                \item
                    \begin{align*}
                        P(\text{both selected}) &= \left( \frac{\binom{2}{2} \cdot \binom{8}{4}}{\binom{10}{6}} \right) \\
                        &= \frac{1 \cdot \frac{8!}{4!4!}}{210} \\
                        &= \frac{70}{210} \\
                        &= \frac{1}{3} \\
                        &= 0.3333
                    \end{align*}
            \end{enumerate}
\end{document}
